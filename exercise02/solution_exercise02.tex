\documentclass{article}
\usepackage{graphicx} % Required for inserting images

\usepackage[portuguese]{babel}
\usepackage[T1]{fontenc}

\title{Teoria da Computação` [EEL881]}
\author{igorcarvalho@poli.ufrj.br}
\date{\today}

\renewcommand{\arraystretch}{1.7}

\begin{document}

\maketitle

\noindent \textbf{Exercício 1)} Calcule o número de Gödel associado à expressão $E: P_2(x_1) \to \alpha_1$

\hspace{0pt}

\noindent \textbf{Solução:} Seguindo os valores associados a cada símbolo nas tabelas 1 e 2, temos

\begin{center}
    \begin{tabular}{ccccccccccc}
        $P_2$ & & ( & & $x_1$ & & ) & & $\to$ & & $\alpha_1$\\
        $\downarrow$ & & $\downarrow$ & & $\downarrow$ & & $\downarrow$ & & $\downarrow$ & & $\downarrow$\\
        $19^3$ & & $11$ & & $17$ & & $13$ & & $3$ & & $17^2$\\
        $2^{19^3}$ & $\times$ & $3^{11}$ & $\times$ & $5^{17}$ & $\times$ & $7^{13}$ & $\times$ & $11^{3}$ & $\times$ & $13^{17^2}$\\
    \end{tabular}
\end{center}

O número de Gödel associado à expressão será o produtos dos primeiros números primos elevados ao valor de cada símbolo

\[
G_E = 2^{6859} \times 3^{11} \times 5^{17} \times 7^{13} \times 11^{3} \times 13^{289}\\
\]

\newpage

\begin{table}[!h]
    \centering
    \begin{tabular}{c|c|c}
        \textbf{símbolo} & \textbf{significado} & \textbf{número}\\
        \hline
        $\lnot$ & negação & 1\\
        $\to$ & se-então & 3\\
        $\exists$ & existe & 5\\
        0 & zero & 7\\
        $suc$ & função sucessor & 9\\
        ( & abre parêntese - separador & 11\\
        ) & fecha parênteses - separador & 13\\
    \end{tabular}
    \caption{Valores dos símbolos}
\end{table}

\hspace{0pt} 

\begin{table}[!h]
    \centering
    \begin{tabular}{c|c|c}
        \textbf{variável} & \textbf{número} & \textbf{exemplo}\\
        \hline
        $x_i$ & i-ésimo primo $>$ 13 & $17, 19, 13, 29, ...$\\
        $\alpha_i$ & i-ésimo primo $>$ 13 ao quadrado & $17^2, 19^2, 13^2, 29^2, ...$\\
        $P_i$ & i-ésimo primo $>$ 13 ao cubo & $17^3, 19^3, 13^3, 29^3, ...$\\
    \end{tabular}
    \caption{Valores das variáveis}
\end{table}

\end{document}
